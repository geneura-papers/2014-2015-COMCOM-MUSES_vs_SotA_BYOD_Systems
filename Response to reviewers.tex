\documentclass[preprint]{elsarticle}
\biboptions{round, numbers}
\usepackage[latin1]{inputenc}
%\usepackage[T1]{fontenc}
%\usepackage{textcomp}
\usepackage{graphicx}
\usepackage{color}
%\usepackage{setspace}
\usepackage{url}
\usepackage[english]{babel}

\begin{document}

%%%%%%%%%%%%%%%%%%%%%%%%%%%%%%%   TITLE   %%%%%%%%%%%%%%%%%%%%%%%%%%%%%%%

\title{Corporate Security Solutions for BYOD:\\ A Novel User-Centric and Self-Adaptive System: Response to Reviewers\' comments}

\section{Editor's comments}
The referees made the following points about this paper.

Editor comments:\begin{enumerate}
\item The presentation/organisation of the paper needs some revision.
\item The authors should cite previous related work and explain the
  additional contribution of the current proposal.
\end{enumerate}

{\emph The paper has been throughly revised. 

\section{Comments by Reviewer \#1}

\begin{verbatim}
The paper describes an ongoing European FP7 research project,
MUSES. It mainly covers the concept behind MUSES, the description of
the architecture, and a high level description of the soft computing
methods to be used, as no actual results have been obtained so far. 

A comparison to existing commercial applications is also included, but
this seems taken from their commercial brochures; it does not offer
any clues on how they differ from the functionality offered by MUSES.
\end{verbatim}

The reviewer is right, the state of the art in commercial solutions was quite brief and descriptive.
We have addressed this issue on the revised version of the paper,
changing several sections in order to compose a State of the art in
applications similar to MUSES, and going deeper in the features that
can be compared with those of our system. 

We have also compared this solutions with the current features of the
MUSES prototype, which is already being used in tests, in Section 5.

\begin{verbatim}
It is also unclear how this paper differs from previous work by the
same authors (and others), which also aimed to describe the system;
more specifically: 
"Enforcing corporate security policies via computational intelligence
techniques", GECCO Comp '14 Proceedings of the 2014 conference
companion on Genetic and evolutionary computation Pages 1245-1252  
\end{verbatim}

The paper has been improved in a number of ways, making it more
distant with respect to our previous works. Moreover, we have remarked
in the introduction the contributions of the present work with
respect to the previous ones, and included also references to other
research whose result has been included in Muses, for instance:

[10] A. Mora, P. De las Cuevas, J. J.  Merelo, Going a step beyond the black and
white lists for url accesses in the enterprise by means of categorical classi-
fiers, in: A. Rosa, J. J. Merelo, J. Filipe (Eds.), ECTA 2014 - Proceedings
of the International Conference on Evolutionary Computation Theory and
Applications, 2014, pp. 125-134.


\begin{verbatim}
In this reviewer's opinion, the authors should wait till they have
results to present. 
\end{verbatim}

Although in this version of the paper we make reference to the release
of the first versions of MUSES, the main objective  of this paper is
to present a state of the art in BYOD and how the issues present in
this area are addressed using computational intelligence techniques.

\begin{verbatim}
The relationship between the scope of the special issue and the
subjects dealt by MUSES is also unclear
\end{verbatim}

Security and privacy are obviously the main concerns of MUSES: data
security and user privacy. The BYOD philosophy tries to maintain data
security while keeping the privacy of the user's own data and
applications, so we think that the current version of the paper are in
scope of the special issue. The new introduction tries to make this
clearer. 


\section{Comments by Reviewer \#2}

\begin{verbatim}

= Global comments =

I'm concerned about the novelty of the information provided in the
document. [REF1] and [REF2] are two already published papers by the
same authors. 

[REF1] A.M. Mora, P. De las Cuevas, J.J. Merelo, S. Zamarripa,
M. Juan, A.I. Esparcia-Alc�zar, M. Burvall, H. Arfwedson, and
Z. Hodaie,  MUSES A corporate user-centric system which applies CI
methods. Presented at the TRECK track at ACM-SAC 2014. 
[REF2] A.M. Mora, P. de las Cuevas, J.J. Merelo, S. Zamarripa,
A.I. Esparcia-Alc�zar, Enforcing Corporate Security Policies via
Computational Intelligence Techniques. Presented at the SecDef
Workshop at ACM-GECCO 2014. July 2014 

Comparing these references with the current article, which includes 21 pages:

- The introduction in page 2 is very similar to the introduction in [REF2]
- Pages 3-4 are the same than section 2 in [REF2]
- Pages 12-15 are strikingly similar to [REF1], even using the same figures.
- Pages 15-19 are a raw copy of [REF2], nearly word by word.
- Pages 19-21 are extremely similar to section 4 in [REF1] and section
5 in [REF2], put together. 

\end{verbatim}

The reviewer is right in part, since some of the contents are the same when a system is being described.
We have followed his/her request and changed/improved several parts in
the text, namely  \begin{itemize}
\item The background has been corrected and expanded.
\item References to papers mentioned by the reviwers and how they
  compare with the current paper (and situation in the project) have
  been added to the introduction.
\item Many more papers have been reviewed and referenced. The
  reference section has grown by half.
\item BYOD solutions have been compared among themselves, and with
  MUSES (section 5).
\item The paper reflects the current, so far unpublished, state of
  Muses architecture and implementation. 
\end{itemize}

\begin{verbatim}
Hence, the only *new* research this reviewer is able to identify is
section 3 (pages 5-11), which includes a very detailed
state-of-the-art of current tools for corporate mobile
security. Unfortunately, this is not really a paper focused on the
state-of-the-art in this field.
\end{verbatim}

As suggested by the reviewer, we have transformed the paper into a
state-of-the-art of systems that offer functionalities that are
similar to MUSES. The paper has been
almost completely rewritten to reflect the state of the art including
changes from the time the first paper was submitted (Samsung dropping
KNOX, for instance). 


\begin{verbatim}
The work is good and based on good ideas, and the fact that it is
already accepted in two conferences proves this. This is why I will
recommended a MAJOR REVIEW of the paper. In mi opinion, the authors
should consider: 

a) extending the state-of-the-art section, and converting the paper in
a "state of the art" or tools for corporate mobile security. They
already provide very detailed descriptions of IBM, Samsung, Sophos and
a few other system, but only a hint about WSO2 and Blackphone, which
are identified as the main competitors of MUSES in section 5. In this
case, sections 4 and 5 are not really needed, but a description of
MUSES as another system like the others could be a good idea. 
\end{verbatim}

Following the reviewer's suggestions, we have transformed the paper
into a state-of-the-art in tools for corporate mobile security. MUSES
has been compared with the existing tools in a more detailed way and
the description of the system has been reduced in the same way. All
sections have been rewritten and new sub-sections have been added to
more accurately reflect the state of the art. 


\begin{verbatim}
b) reducing the length of the state-of-the-art section, currently too
long in my opinion, and adding lots of new information and
descriptions of new work, research and results to the sections 4 and
5. 
\end{verbatim}

We have found more interesting to follow the first suggestion, for
which we are grateful, leaving this for future work.

\begin{verbatim}
My preferences are for option b). In case the authors are interested,
they will find next specific comments for a major revision of the
work.
\end{verbatim}

Option {\em a)} was actually closer to our initial intention when
submitting this paper, so we opted for it. 

\begin{verbatim}
= Specific comments =

According to page 12, line 7 "the user interacts with the devices
(...) through the MUSES graphical interface inside his or her own
context". This is surprising. I have pictured MUSES as a hyper-visor
module but, according to this sentence, MUSES is a GUI that contains
every application the user has access to. I think further
clarification are needed about how the user interacts with his
applications and what this GUI the authors are describing is
exactly. Contexts in this sentence "(situation, connection, status)"
do not match contexts in figure 2 "(social, economical, political)".
\end{verbatim}

This observation is very much appreciated since it is true that what was written could lead to a missunderstanding. The description has been changed, and the context properly defined and explained.

\begin{verbatim}
Page 13. The first reason to use a client/server architecture is that
the event correlation and self-adaptation processes need a very
powerful machine (a server), but this statement seems not be based on
any real experiment run by the authors or any previous references. In
addition, I find the second reason to user a client/server
architecture quite surprising. Quoting: "there are two clearly
separated parts in the system, namely the users (clients) and the
enterprise (server)" I think this is a very weak justification. I can
identify many other actors in the system: data providers (cloud
system, databases...), other workers in the enterprise (for corporate
applications), the friends of the user (for private applications),
external application providers... Hence, if the "second reason" is of
any real value, MUSES would have a P2P architecture. From my point of
view, the reasons given by the authors are too week and, in any case,
there is not really any necessity to justify 
why MUSES has a client/server architecture.
\end{verbatim}

The reviewer is right. We have rewritten the paragraph justifying the
use of a server machine due to its computational power, needed for
applying Data Mining and Machine Learning techniques over Big
Datasets, and also to centralise the data gathering of the system. We
have removed the text explaining the consideration of two parts in the
system. 

\begin{verbatim}
Page 13, line 45 says all data in MUSES will be encrypted. It is not
clear if the authors mean "the Decision Table and the even logger", or
"all data in the devices, including personal data, contact list,
emails..." If it is the latter, authors should explain how MUSES
enforces this policy. Does it control all applications in the mobile
phone? Does the mobile phone run only white-listed applications? 
\end{verbatim}

We strongly agree with this and a detailed description of the encryption methods, that the second prototype of MUSES will implement, has been included.

\begin{verbatim}
Some specific examples about the kind of user activity MUSES monitors
and the kind of actuators MUSES implements will be welcome. As is,
sensors and actuators seem like a general idea and readers will
understand the system better with specific examples.
\end{verbatim}

% Not done  so far

\begin{verbatim}
Page 14, line 47 identifies the MusCRTEP as the core of the MUSES
system and provides two references. Unfortunately, these references
are not the implementation of MusCRTEP but a general definition of
what a decision system does. These references are not really needed
and a much more detailed description of MusCRTEP would be welcome
since even the authors identify this module as the core of the
system. 
\end{verbatim}


The description of the performance of the CRTEP has been extended
and improved, with references to two Project Reports which better fit
to the contents. 

\begin{verbatim}
Page 18, lines 20-37 hint about the importance of privacy enhancing
technologies in a system that monitors the activity of the user and
sends this activity to a centralized server to be "anonymously
managed". Further information about how the system manages this
information is needed (does private information ever leaves the
personal devices? Which activity is monitored and collected?), as well
as an explanation about how these data is anonymized. Is it anonymized
at the mobile phone? Or in the server's database? 
\end{verbatim}

 The way user data is anonymised in the system, as long as other legal
 aspects about the data privacy, have been properly detailed, as well
 as complemented with a reference.  %where and which one?

\end{document}
