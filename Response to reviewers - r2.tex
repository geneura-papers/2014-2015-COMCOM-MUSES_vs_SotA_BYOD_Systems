\documentclass[preprint]{elsarticle}
\biboptions{round, numbers}
\usepackage[latin1]{inputenc}
%\usepackage[T1]{fontenc}
%\usepackage{textcomp}
\usepackage{graphicx}
\usepackage{color}
%\usepackage{setspace}
\usepackage{url}
\usepackage[english]{babel}

\begin{document}

%%%%%%%%%%%%%%%%%%%%%%%%%%%%%%%   TITLE   %%%%%%%%%%%%%%%%%%%%%%%%%%%%%%%

\title{Corporate Security Solutions for BYOD:\\ A Novel User-Centric and Self-Adaptive System: Response to Reviewers' comments}

\maketitle 

\noindent
Dear Sirs,\\

We really appreciate the opportunity for improving our work. Following the reviewers' requests and advices we have improved the paper, incorporating all suggestions.  

In the next paragraphs you can find these comments and how they have been addressed, along with the changes done.\\

\noindent
Yours sincerely,\\
The authors.


\section{Editor's comments}

\begin{verbatim}
Dear Dr. Mora,

Reviewers' comments on your work have now been received. After
analyzing the reviewers' reports and the manuscript, the Guest 
Editors are recommending that you perform a further MAJOR revision 
of your manuscript to address the reviewers' comments.

While revising the manuscript please carefully revise the analysis 
of the state of the art to include all recent and relevant works 
in the field.
An updated analysis of the state of the art is mandatory for a 
publication in Computer Communications.

If you decide to revise the work, please submit a list of changes 
or a rebuttal against each point which is being raised when you 
submit the revised manuscript.

\end{verbatim}

We are very grateful for the opportunity of improving the paper in this revised version. 

Following the Editor's and some of the reviewers' requests, we have updated the state of the art, revising and commenting in the text (Section 2) several works with common topics with this manuscript. The new references can be consulted b the end of this letter.

In addition to this, we have addressed all the requests and suggestions by the reviewers. The responses are included below in this document.

% -----------------------------------------------------------------------------

\section{Comments by Reviewer \#2}

% Es quotation, no verbatim, �no?
% Antonio - con quotation no se distingu�a bien el texto del revisor y el de la respuesta. S�lo hac�a un sangrado. Creo que as� se ve mejor.

\begin{verbatim}
After reviewing the new version of the document, I admit the authors
have  substantially changed section 3 (state of the art) and they have
improved and extended the introduction and sections 4 (description of
the system) and section 5 (comparison with other systems) I
acknowledge the ideas are really interesting and exciting, but I'm
still not sure if the authors are providing new results not published
in their previous works or just rephrasing their text. 

Since there is an obvious over-the-top quality in their work and some
good ideas, I'll let the editors decide if the contents are novel
enough (compared to other previous works by the same authors) to be
published in the journal.
\end{verbatim}

We would like to thank the reviewer for his/her opinion and would lead him/her to have another look at the new version, since it has been highly improved.  
As a summary:
%
\begin{itemize}
\item We have described in a more detailed way our previously published papers in this scope (in Section 1), clarifying the differences and contributions of this work with respect to them.
\item Section 2 has been deeply improved adding a much more complete state of the art in the scope of this work. To this end, several research works have been read and commented in that section and along the text. This has helped us to better compare our system against the other products available in the market. Also, an analysis of the best practices for BYOD have been applied for each product and our system.
% Antonio - TODO: Fergu, completa esto t� que tienes la idea global de lo nuevo en el SotA, por favor. FERGU: Hecho. Metido tambi�n lo de las best practices
New references added after the review can be found at the end of this letter.

\item Section 3 has been rewritten in a big part, including a new taxonomy for the analyzed systems at the beginning of the section. This taxonomy has been created taking into account the expertise extracted from the new references of the state of the art, such as the best practices by Romer \cite{Romer14BestPractices} or the hands-off/hands-on classification by Scarfo et al. \cite{Scarfo12survey}, among others. Furthermore, some new tools have been added, and all the products have been better classified considering the defined taxonomy.
% Antonio - TODO: extender esto de la secci�n 3 y 5 un poco m�s. FERGU: Done

\item The description of MUSES system has been also improved. Several points have been clarified and better explained in Section 4. Moreover, a prototype of the system has been tested in an actual company with success, according to the obtained results. This is explained in the new Section 4.4 (page XX), where the results of some trials are presented.

\item Section 5 has been also completed and enhanced with respect to previous version. It includes a better comparison between MUSES and the rest of tools. Two new tables (1 and 2) have been added summarizing the main features of the systems, for the sake of clarity.

\item A complete proof-editing process has been done, reviewing the English grammar and rewriting several paragraphs for a better comprehension and flow.

\end{itemize}
% Antonio - TODO: A�adir m�s cosas (medianamente relevantes) que se hayan hecho

These changes are better described in the responses to the others reviewers, so we strongly suggest this reviewer to read them in order to have a better idea of their impact in the new version of the manuscript.

% -----------------------------------------------------------------------------

\section{Comments by Reviewer \#3}

\begin{verbatim}
This paper describes MUSES, an adaptable framework for enforcing 
security policies on mobile devices. This tool relies on Machine 
Learning and CI techniques to predict policy violations and is 
self-adaptive on user behaviors.
The idea seems interesting although several technical concerns 
arise from the reading of this work:
1.      The tools aims at enforcing security policies, however 
the security response occurs only AFTER a particular pattern is 
recognized to be potentially malicious. From my understandings, 
there is no a-priori enforcement except for the prediction based 
on event logs. This could mean that a security violation could 
in principle happen and only later recognized. This is not so 
suitable for a corporate environment with strict security 
requirements. This statement should be discussed and preferably 
discredited with experimental results.
\end{verbatim}

MUSES starts from a defined set of security rules that the CSO must
define. Ideally, these rules should deal, in principle, with all the
potential security incidents that the company aims to detect. MUSES
would be able to define new rules, by means of refinement, which could
be labeled, i.e. assigned a decision, after the corresponding events
have happened. However, it can also infer or create new rules, which
the CSO must approve, using computational intelligence
techniques. These rules could deal with unexpected situations, which
have not happened previously . 

Anyway, the idea is to define a set of `restrictive' rules at the very beginning, able to ensure the system safety in a big amount of situations. Then, MUSES will refine or improve this set, making more specific and optimal rules.

Moreover, MUSES offers visual analysis tools to the CSOs, which can
aid or guide them to detect suspicious or anomalous patterns, and to
study how to avoid these patterns becoming malicious to the company. 

In order to clarify this we have rewritten and completed the second
paragraph in section 4.1 (page XX). It is now: 

\textit{The client is an important part of the a-priori risk treatment, because it both monitors the whole context, and finally acts or warns the user in event of a security violation. 
But security violations are mainly identified by security rules, which is why MUSES needs at least an initial set of rules, defined by the CSO. Then, it makes use of DM and CI techniques to improve or optimize them, enhancing the coverage. 
That is to say, the more sensors are implemented, the more information related to event context can be extracted. Implementing a sensor means to include a programmed method which triggers MUSES to start `listening' to some parts of a device. Then, the initial security rules establish which situations might cause a security violation, and the sensors provide with information which helps the MUSES server to identify events that are alike to the ones that are already known as dangerous.}

We have also improved this clarification in section 5 (page XX), adding the text:

\textit{However, MUSES presents a limitation regarding the enhancement of rules, since, in principle, it cannot predict (generate rules for dealing with) unexpected or unknown events which could lead to a security incident. The philosophy is that the initial set of rules, defined by the CSO, should be very restrictive regarding possible unexpected users' behaviors or events, in order to avoid as much security incidents as possible. As the system works, MUSES would be able to define new rules (through refinement) which could get an associated decision after the corresponding events have happened. Thus, this will lead to get eventually an optimal set of security rules.}

\textit{In addition, MUSES can also infer or create new rules using computational intelligence techniques. These rules could deal with unexpected situations (not previously happened), but must be previously approved by the CSO. Of course, everything is constrained by the available set of sensors which, in turn, define the possible information that MUSES will analyze and use in the refinement and inference processes.}

\begin{verbatim}
2.      A strong limitation is that an effective enforcement is 
only possible with MUSES Aware Apps. It is unclear the effectiveness 
of the approach with commercial applications. In page 14 "MUSES 
could just detect very specific events and also do very limited 
actions". With those assumptions, it seems unlikely that a company 
would adopt MUSES since it is not able to offer an adequate level of 
security. I think that this is very far from being acceptable for a 
BYOD solution. Also in this respect, experimental results are needed.
\end{verbatim}

% [Pedro] traducido e incluido a continuaci�n => \textcolor{blue}{TODO - Justificar su val�a. Las empresas querr�n MUSES y querr�n que sus empleados utilicen MUSES-Aware Apps porque les garantizar�n un control sobre lo que se hace en la empresa y los posibles incidentes de seguridad.}

Due to its functionality, proposed system might be of interest for many companies as using MUSES-aware Apps they could control any operation that a person can do with their mobile or desktop devices.

With respect to the experimental results, a new section (4.4, in page XX) has been included describing some trials conducted in an actual company, using a prototype of MUSES on Android devices. The results show the effectiveness of this system, which has roughly reduced from 23000 security violations in the first stages to 0 in a few weeks.

% Antonio - completar algo m�s, si alguien quiere


\begin{verbatim}
3.      MUSES Sensors are essentially user-space applications that 
collects logs on specific activities. This aspect really concerns 
me since a compromised device or compromised sensor apps could vanish 
all MUSES effort. Authors should deeply discuss this aspect.
\end{verbatim}

To clarify this point, a more detailed definition and description of the sensors has been included in Section 4.1 (page XX). In addition, a discussion on how the case the reviewer mentions could affect the system has been added just below the list of sensors, as:

\textit{It is important to note that MUSES does not work as an antivirus itself, but has a sensor which checks if the device has already an antivirus, and raises a security violation alert if not. It might happen then that the user ignores this advice and the device becomes compromised. As it would be explained in the next section, the sensors are constantly sending their monitored values, and therefore patterns are created, stored, and processed on the server. In case the device gets infected with rootkit-type malware (53), a sudden change in the values of the sensors will mean a significant change in the patterns. Then, this change in the patterns will be noticed by the server, which will immediately warn the user, and decrease the device trust value.}

\begin{verbatim}
4.      Paper structure is really confusing. Many essential concepts 
are introduced and explained too late in the paper forcing the reader 
to go back and forth to understand the architecture (e.g. MUSES aware 
apps and Data Mining techniques). I suggest authors to deeply revise 
the presentation of their work.
\end{verbatim}

Following the reviewer's suggestion, the paper structure has been redefined, with an introduction that presents the work, a section that introduces preliminary concepts followed by the state of the art in corporate mobile security and, eventually, the description of the framework that is the main point of the paper. Changes have been too extensive to include here, but great
care has been taken to talk about concepts only when defined
previously. For instance, the concept of MUSES-Aware apps is now
introduced in the context of the API in Section 3.5 and only used
later on, from Section 3.6.4, in the context of other tool. The
concept of data mining is presented along with machine learning in section 4.3.1. 
We have also added some tables (in Section 5) as an additional summary of information, for the sake of clarity. 

% Antonio - Mejorar un poco esta justificaci�n, si hay tiempo

\begin{verbatim}
5.      A discussion on research activities on BYOD is completely 
missing. This is unacceptable for a solution that aims at solving 
BYOD security issues. Just to cite a few:
a.      Gessner, Dennis, et al. "Towards a User-Friendly 
Security-Enhancing BYOD Solution." NEC Technical Journal 
7.3 (2013): 113.
b.      Samaras, Vasileios. A BYOD Enterprise Security Architecture 
for accessing SaaS cloud services. Diss. TU Delft, Delft 
University of Technology, 2013.
c.      Armando, Alessandro, et al. "Enabling BYOD through secure 
meta-market." Proceedings of the 2014 ACM conference on Security 
and privacy in wireless and mobile networks. ACM, 2014.
\end{verbatim}

The reviewer is right. Although several works have been cited in other sections of the work, a proper discussion on research on BYOD was necessary. Therefore, we have analysed the papers suggested by the reviewer, along with several new relevant ones in Section 2 (page XX). These new references have been also cited in other sections of the work in order to justify MUSES features; for example, the set of best practices for BYOD security proposed by H. Romer. 

These works have been also considered to propose the new taxonomy included in the paper (beginning of Section 3, page XX). These references are \cite{Garba15organisational,Scarfo12survey,Gessner13userfriendly,Armando14metamarket,Haejung12Door,Samaras13SaaS,Miller12Privacy,Romer14BestPractices}, and can be found at the end of this document.

\begin{verbatim}
6.      The experimental section is completely missing. Is MUSES 
able to identify security violations? Which are the computational 
cost for both the company (server) and users (mobile devices)? Are 
there any limitations?
\end{verbatim}

We agree with what the reviewer remarks, and so we have created a new Section (4.4, in page XX), describing the trials that have been conducted with the first prototype of MUSES. In that section there are commented the ratio of security violations detected and how the system has influenced in the complete reduction of this number to 0.
This has been also commented in the Introduction of the paper. 

The main limitation of MUSES, also commented in Section 4.4, is the need of an initial set of defined security rules to start working. The other limitation comes from the set of deployed sensors, which are the only input of data in the system, so it cannot manage security violations happened by different ways.

To avoid this in part, there is a set of guidelines for the companies to build adequate security policies. For instance, the set of initially defined rules used for the trial included a blacklist of forbidden applications, a whitelist of allowed ones, and requirement of antivirus, among others.

With respect to the computational costs, in this first prototype they are around 0.3 - 0.5 seconds in the client side, if there is already a decision in the local table for the event. If there is need to ask the server to make a decision (by means of Event Correlation + Risk and Trust Analysis), the time grows up to 2 - 2.5 seconds in the worst case.
This has been also commented in Section 4.4.

% Antonio - completo con cifras que me ha dado Sergio.


\begin{verbatim}
Under such considerations I recommend authors for a *STRONG* major 
revision to cover the aforementioned points.
\end{verbatim}

The work has been deeply improved addressing all the reviewer's suggestions. We would like to thank he/she for his/her careful revision and interesting comments.


% -----------------------------------------------------------------------------

\section{Comments by Reviewer \#4}

\begin{verbatim}
In this work authors present state-of-the art in BYOD solutions, 
and introduce a new solution, MUSES, a result of an EU FP7 project. 

General and specific comments:

The abstract of the paper should be improved. Sentence on page 1,
lines 23-25 is unclear. It suggests that the BYOD solutions appeared
in the market with the purpose to adapt to it in a secure way. 
\end{verbatim}

The abstract has been changed and that confusing sentence has been
eliminated. Now we think it is clear, as intended, the motivation,
context and focus of the paper. 

\begin{verbatim}
The description of IBM products, in section 3.1, does not provide
clear distinction between various service types provided by IBM. It
furthermore lacks focused coverage of services relevant for this
work. 

In analysis of IBM offer, authors refer to general web site for mobile
security products and consumer guide/whitepaper that explores the
general problem of BYOD. They consider services that are not purely
related to BYOD and MUSES. For instance, it is not clear how
"Enterprise Wireless Networks" (page 6, lines 21-26) fit into
comparison, and how establishment of secure connections can be related
to MUSES's adaptability. On the other hand, the products such as
MobileFirst and MaaS360 are omitted. The suggestion in this regard is
to consider rewriting this part, including updated references and
current tools in comparison. 
\end{verbatim}

Following reviewer's suggestions, this section has been rewritten, and
the references mentioned by the referee, including new products, have
been considered in the study. The different IBM tools and services have been described 
(end of Section 3.1, in page XX), and compared to MUSES in Section 5.
%FERGU: PONED ESTO MÁS BONICO!
% Antonio - ahora est� un poco mejor, pero no s� si m�s bonico. Fergu debes justificar el por qu� se han puesto los tres servicios de IBM, ya que el revisor dice que algunos no est�n relacionados con el tema del trabajo... t� que sabes de esto de IBM. ;D
% Fergu, tambi�n falta la comparativa con MUSES de esas herramientas (Secci�n 5)...
% [Pedro] he cambiado un poco la redacci�n   ;)
% FERGU: Bueno, al final estos productos forman parte del "Entorno IBM" en general, que s� se ha comparado con MUSES. Paloma ha reescrito tambi�n la secci�n y quitado lo que sobraba.

\begin{verbatim}
In section 3.3 (p.7, sentence in l.18-19) there is provided a
statement about success of two solutions, however without referencing
relevant sources or providing details. It is furthermore questionable
whether this statement is relevant for the scope of this work at all.
\end{verbatim}

We agree with the reviewer and we have re-formulated this statement, as what actually happened was that these two tools had many problems. We explained why in the previous version of this work. Also, we have included new aspects of these tools, in order to classify them in the newly added taxonomy in section 3. Therefore:

Even though these two tools are provided by two different manufacturers, Samsung and Blackberry, they both have in common their situation, as they found problems in the market in spite of their features. On the one hand, Blackberry sales have been decreasing since 2010 \cite{Blackberry_sales}.
On the other hand, Samsung revealed at the Barcelona Mobile World Congress 2013 the KNOX application \cite{Samsung_mwc13}, which was expected to be available for its last Galaxy smartphone generation. Then, it was delayed and again presented in the same congress in 2014 \cite{Samsung_mwc14}, but still remained delayed until it was apparently found insecure \cite{Samsung_insecure}. Finally, Samsung decided to collaborate with Google in Android Work \cite{Samsung_android}, as mentioned in Section \ref{subsec:androidwork}.

With regard to their features, Samsung KNOX, as well as Blackberry Balance, are more focused on the device side, while the solutions mentioned in previous subsections were more centered in being tools for CSOs. This also means that they are not device-independent, as KNOX is only for Samsung devices and Balance is integrated in BlackBerry 10 \cite{Blackberry_tool}. Looking at their situation and problems, it could be thought that making a BYOD solution platform-dependent is counteractive, which is why MUSES is independent of the device platform.

\begin{verbatim}
Section 3.6.1 covers WSO2, an open source tool that should be freely
available for evaluation and comparison. However, in review, this tool
has been only roughly mentioned. The context and relevance to the work
are omitted, as well as features and comparison with MUSES. It is not
clear therefore how this tool and MUSES differ. 
\end{verbatim}

The reviewer is right and it is true that WSO2 is a better competitor for MUSES than other tools, as is open source and multiplatform. However, it does not include offline working mode neither is able to refine policies. We have explained this in its section:

Although this tool also includes features such as device location, remote wipe, or encrypt storage, its main disadvantage is that it does not work in offline mode. In the documentation \footnote{\url{https://docs.wso2.com/display/EMM200/Working+with+Policies}}, it is clearly stated that the policy compliance can be monitored while the devices are connected to the WSO2 EMM server.

\begin{verbatim}
Section 3.6.4 (p10, 44) states "are Worx-enable are capable". The 
meaning of this part is not clear, nor to which concept Worx refers.
\end{verbatim}

We have made clear that Worx is an SDK, which is a development environment, and also completed the explanation of ``Worx-enable application'' by comparing it with the concept of MUSES Aware applications.

XenMobile with the development environment called Worx App SDK, made by Citrix Systems, provides BYOD security services to companies using fine-grained policies to prevent users from performing unallowed actions \cite{WorxSDK}. For example using the mobile phone camera, GPS, or microphone. These policies can be turned on or off using its own GUI. Citrix is framework-enabled, and it is aware of the apps installed on the device. All the apps that are Worx-enabled are capable of interacting, and thus offering the user a better experience. This concept is similar to the `MUSES Aware' one, which was mentioned earlier in this section, and consists of adapting the applications to allow communications with the BYOD tools.

\begin{verbatim}
Basically, in section 3 the following possibilities for improvement 
are identified:

1) Explain why particular solutions are chosen for inclusion in review 
and what is their relevance? Did authors consider other (possible) 
solutions relying on DM and CI?
\end{verbatim}

We understand that this might needed some clarification, so we included two references to web articles. First, to a top 10 of BYOD applications, the most recent we have found, and yet it is from 2013. As consequence, we found that some of the reviewed tools in the website ceased to exist or were acquired by other companies. As the other tools concern, we considered the most important manufacturers, companies, OSs, and devices, with regard to their sales.

The products that have been reviewed were chosen either because they have appeared in top-10 web articles \footnote{\url{http://www.networkworld.com/article/2357899/data-center/108866-10-Mobile-Device-Management-Leaders-That-Help-IT-Control-BYOD.html#slide11}}, or they have been developed for the main smartphone platforms \footnote{\url{http://www.idc.com/prodserv/smartphone-os-market-share.jsp}}. Actually, the selection process have been difficult, as some of the initially found tools or companies in the web were bought by others that we do review here. Also, because this market is still growing, there are not many solutions to review, and here we display the ones more alike to MUSES. This section introduces the most relevant features of these products, as they can be considered related to the MUSES objectives.

\begin{verbatim}
 2) Considering variety of BYOD tools and their different features, 
authors might consider to introduce or reuse a taxonomy or 
classification of those approaches, and state the place of MUSES 
there (or define a new subcategory, if deemed necessary)
\end{verbatim}

Following the reviewer's suggestion we have introduced a taxonomy for BYOD, according to the works that we have revised in the state of the art (Section 2). Different concepts mentioned in this improved state of the art, such as the hands-off/hand-off differentiation or BYOD best practices, have been taken into account to justify the proposed taxonomy. This taxonomy is described at the beginning of Section 3 (second paragraph, page XX). Then, all the tools, including MUSES, have been classified in the types defined by the taxonomy, and summarized in Table 1.
% Antonio - completar con el n�mero de p�gina final
% Antonio - TODO: decir si MUSES y el resto de sistemas se han encuadrado dentro de esa taxonom�a
% FERGU: completado lo de la taxonomia (falta lo de los n�meros de p�gina a�n!) TODO

\begin{verbatim}
 3) Some presented solutions miss their relationship to MUSES 
(coverage, features, comparison). This point also refers to 
inconsistent coverage amongst solutions and weak arguments 
favouring MUSES. Either coverage should be more consistent by 
focusing on common (present or missing) aspects, or it should 
be elaborated why such coverage is missing. Also see the comment 
for discussion part bellow.
\end{verbatim}

% [Pedro] traducido e incluido =>  \textcolor{blue}{TODO - justificar que todos los sistemas no tienen las mismas propiedades, pero s� la misma filosof�a}

Although not every available system offer the same functionality, and even not all are free, they follow the same philosophy.

\begin{verbatim}
On p.12, l.9-12 authors state that MUSES can be installed in every 
mobile or portable device, independently of the operating system 
and device type. This is also derived from the title "Multiplatform 
Usable Endpoint Security System".
The authors however do not provide additional details on how is this 
achieved in the practice and are there any limitations in regards to 
that. How they managed to provide the same system for Android, Windows, 
iOS, PCs, without limitations? How MUSES manage the variety of OS 
versions and underlying platform features provided to the app? This 
feature is important differentiator against some other solutions and 
in opinion of this reviewer should be elaborated in more detail.
\end{verbatim}

We agree and therefore we have included in section 4 that MUSES has been developed for Android, Windows PCs and phones, and that an iOS client is being developed. The same sensors are available in Android and Windows, but there is a strong limitation with monitoring iOS devices if they were not been \textit{jailbroken}. This means that if an iOS device has not been rooted, MUSES could not monitor every proccess in the device, and also it would have access only to some sensors. However, to be rooted is one of the situations that MUSES tries to avoid, which is why it depends on MUSES Aware apps. Thus:

MUSES client is available up to date to be used in Android devices as well as in Windows mobile devices and PCs; the iOS client is under development. The same sensors are available in Android and Windows, but there is a strong limitation with monitoring iOS devices if they were not been \textit{jailbroken}. This means that if an iOS device has not been rooted, MUSES could not monitor every proccess in the device, and also it would have access only to some sensors. However, to be rooted is one of the situations that MUSES tries to avoid, which is why it depends on MUSES Aware apps. As far as the server is concerned, it has been developed using Java to be installed with an Apache Tomcat in the corporate
server, independently of its OS. Both sides are connected through a secure channel using HTTPS over the Internet.

\begin{verbatim}
When stating (p.13, l.15-24) that MUSES relies on AES based 
symmetric encryption, does it mean that information flows in 
MUSES client internally, as well as client-server communication, 
are encrypted with the same key? Does MUSES uses client specific 
and/or temporal keys? What measures are taken to protect this key 
from malicious applications both on client and server side? How 
and where is this key generated? What is lifetime of the key? In 
my opinion this part should be reworked, providing more detailed 
description of security aspects of their approach. Also please 
see security related comment bellow.
\end{verbatim}

We agree with the reviewer about this paragraph not being clear, which is why it has been rewritten and clarified. At the end of the introduction of Section 4 (page XX) we have improved the explanation about MUSES data security and authentication, and added a new reference \cite{blasing2010android}.

The MUSES system also performs an authentication process when the user logs in. The MUSES server authenticates to the MUSES client by means of a self-signed certificate. The authentication of the users is based on Spring Security, which is a Java EE framework that provides authentication, authorization and other security features for enterprise applications. When the user connects to the MUSES server for the first time, the MUSES client asks the server for the cerfiticate, wich is sent and installed in the client local database if the user credentials are right. As the user credentials are also stored in the MUSES client local database, the user can also log in to MUSES, and it therefore can apply the Device Policies. With regard to internal data security, MUSES makes use of the advantages that the OSs themselves offer to developers. More concretely, the first prototype for Android includes functionalities provided by the Android Application Sandbox (51). This means that other applications cannot access MUSES data, nor even other developers. The only way to do this is by rooting the device, which is a state that MUSES can detect, and therefore warn about its implications. With regard to the encryption of the rest of the device data, MUSES stands for having a security policy which advices about the installation of existing open source tools \footnote{For instance, \url{https://github.com/neurodroid/cryptonite/blob/master/README.md}}, instead of having an own implementation.

\begin{verbatim}
On p.14, l.16-28 authors introduce the concepts of MUSES Aware 
App and MUSES API. In my opinion they should provide more details 
on integration options and necessary effort to make an application 
MUSES Aware. This is a crucial aspect of their approach as this 
requirement and related complexity might hinder the adoption of 
the solution.
\end{verbatim}

We strongly agree and therefore we extended the explanation of the particular concept of MUSES API, in Section 4.1. We agree that this was necessary because the MUSES API is not a typical API which is used for developing part of third party applications. Instead, the MUSES API provides the necessary resources to make those third party applications able to communicate and deeply interact with MUSES. Then:

On the one hand, the \textit{MUSES Aware App} is an application adapted to MUSES, so the system can directly interact with it. This application must be implemented using the MUSES API (Application Program Interface). It should be noted that MUSES is an application that is running in the background, and third party apps can communicate with MUSES and ask for a response. Then, the MUSES API is not a normal API for building the third party application, but a collection of methods to allow this communication\footnote{The MUSES API is defined in the project, so for every application desired to be MUSES Aware, it should be implemented using this.}.
The main advantage of having MUSES Aware applications, is because actuator is able to actually forbid the users for doing something, if it implies a big risk for the company.

\begin{verbatim}
On p.17, l.7-9 authors state that the data sets can be huge, 
leading to necessity to apply Big Data processing methods. They 
however do not provide at least an indication of the data 
amount/complexity that is gathered and processed by the system, 
and at what rate it grows. Is it really necessary to employ 
Big Data processing?
\end{verbatim}

The justification for using Big Data techniques has been completed in section 4.3.1 with results of the first MUSES prototype trials. Then, these results have been extrapolated to an example case of a medium sized company, and by considering 6 months as the maximum period for storing the data. This way:

The set of data is composed by the so-called patterns, which consist of an occurred event and all the related information to that event. This means that the context in which the event was, is translated into a number of \textit{attributes} of the pattern. Moreover, a lot of attributes are extracted from the events, for obtaining the more information as possible. 
During the first trials, a mean of 201 events per day and user was observed. By considering just a medium size company, of 250 employees, that would make an average of 50300 events per day, of which a high number of features are extracted. Furthermore, if for instance 6 months are considered for the `hard limit' (see previous section), the amount of events to be processed would exponentially grow, until reaching the 9 millon events. For even bigger companies, this all result in huge datasets, so that MUSES makes use of several Big Data processing methods (48), such as: [...]

\begin{verbatim}
On p.18, for two bullets announced under l.46-47 there is not 
clear if they are included in the refinement process or not. 
In the later case, it is suggested to explain why they are not 
present in the processing. Maybe these bullets could be moved 
into part relating to further work.
\end{verbatim}

Agreeing with the reviewer, the refinement process inside MUSES has been explained with more detail in Section 4.3. Also, it has been included how the three Evolutionary Computation techniques depend, or not, one from each other. We agree that this is mandatory because is one of the most important features of the system and should be completely clear for the readers. Additionally, the concept of ``server graphical interface'' has been introduced and explained, as it is a tool for the CSO to interact with the system, define rules, consult the system status, and control the rule refinement, among other useful features. Thus:

Another important fact is that MUSES functionalities are completed by a human controller, normally the CSO, who supervises the system activity by means of a graphic interface they can interact with. Thus, adapted and inferred security rules are not directly added to the current set of rules, but they are proposed instead as draft rules to this human controller, in order to be accepted if they are interesting and useful. The system also uses the acceptance or rejection of new rules as `feedback`, learning from these decisions, and refining the rules according to this.

In addition, we added how every information that the Knowledge Refinement System module manages is used by every technique:

As the rule refinement process is considered of the same importance as inferring new ones, the following three approaches have been implemented:

\begin{itemize}

\item A \textit{GP rule inference} method, which generates new rules in order to `cover' those situations that have been not contemplated in the current set of rules. Thus, a new rule would be created for dealing with the patterns to which the classifier could not assign a class.
The generation of new rules is done by considering the so-called \textit{dictionary}, i.e. a set of terms corresponding to all the possible context situations and user actions in the system, which are the antecedents and consequents \footnote{The antecedents are the conditions of a rule that have to be met to apply the consequents.} of the security rules to be inferred.
The evaluation of these rules is done by considering the information that has been described. Thus, it is possible to `simulate' the whole system behavior when the new rule is included. Therefore, the system gets a value of its performance in terms of, for instance, how many of the events that were not classified or misclassified, are covered now. As was also mentioned, the user or device trust values influence the creation of the rules.
Finally, the inferred rules are proposed to the CSO as ``new, but not yet refined'', along with their evaluated performance.

\item A \textit{GP rule refinement} approach, which optimizes the current set of rules, adjusting the values in the conditions. The set of rules that could be refined is composed by the new rules from the GP-based inference, but also by the original set of rules which was already completed by the Data Miner. Thus, some superfluous parts of the rules and even complete rules could be removed or improved, obtaining for instance specializations or generalizations of existing rules which could lead to a better performance.
The evaluation of the whole set of security rules is done by considering the number of unlabelled patterns that are `covered' after the adjustments. Again, the rules at the end of this step are presented to the CSO as ``refined''. The CSO is the one who decides, according to the performance simulation results, if an inferred rule, or the refined one, is the rule which is finally included in the system.

\item A \textit{GA optimization} algorithm for adapting the values of the assets. These are numerical representations of the importance of the corporate assets, and are considered in the Real-Time Risk and Trust Analysis process, in order to assign a risk value to every potential decision that can be made by the system.
By evaluating the partial solutions proposed by the GA, this approach provides the CSO, who is in charge of assigning and adjusting these values over time, with information about the most dangerous situations for certain assets. In this process, then, the context information that is gathered by the system is very important.
The adaptation or adjustment concerns the change in the value that an asset has due to a loss of importance, for instance, for being an out-to-date document.

\end{itemize}

As a summary, the Knowledge Compiler module uses two EC-based processes for inferring and refining rules, plus another one for adjusting values of the assets. Both the inferred rules and the refined rules are presented to the CSO, who includes them or not in the system, and this acceptance or rejection acts itself as a feedback for future rule inference.

\begin{verbatim}
From the presented concepts behind MUSES and its flows, it
is not completely clear how it approaches "newly discovered 
vulnerabilities or threats" (p.20 l.27-28). It is hence 
suggested to clarify this statement and provide explanation 
to which extent is rule refinement and adaptation used to 
discover and address new vulnerabilities and threats in the 
software and systems.
\end{verbatim}

We agree with this, so the way MUSES creates new rules through classification and also refinement, has been better explained in Section 4.3. Also, in section 5 this has been remarked, as well as completed by explaining that the rule creation process is aimed by a risk and trust analysis in real-time. The reviewer can consult the test in the response to the previous statement. Also, of Section 5:

Also related to this issue, a very big advantage of the MUSES system that is not present in the other solutions is its self-adaptivity power. The proposed system uses different methods to create new security rules, being their aim to cover new vulnerabilities or threats. Thus, MUSES is able to adapt to changes by applying classification techniques to create new rules, and then refining the whole set of existing policies. Additionally, MUSES is able to discover new threats by a combination of real-time risk and trust analysis plus a classifier trained with all the occurred events in the system.

\begin{verbatim}
Similar applies to human-computer interaction (p.20 l.34). 
Although some details are provided in l.43-45 on the same page, 
in opinion of this reviewer, the statement about progress beyond 
the state of the art should be backed with more (stronger) details 
and data.
\end{verbatim}

We agree that it was not clear how MUSES contributes to the State of The Art about HCI. Therefore, it has been explained in section 5 how MUSES tries to have an influence in the user bevahiour through the warnings, as long as securing by adaptation of the security rules. Then:

As far as the HCI is concerned, in Section 2 we talked about how it has been demonstrated by Shaw et al. (21) and Herath et al. (22) that the compliance of the employees of a company, with respect to the security policies, increases by educating them and decreases by punishing them. Therefore, the fact that MUSES is focused in avoiding security incidents due to users' unawareness of the ISPs, it sets up a significant advance in the state of the art. This is because not only the incident is avoided, but the user is educated, which at the same time avoids future risky situations.
With regard to device monitoring, MUSES  also take into account the so-called context observation, by which private or professional scenarios can be detected, or predicted, based on advanced machine learning techniques. Finally, the project is concerned about legal compliance in regards of Information Security Policies, so that it  contributes to the proposal for the EU Data Protection: legal binding force and legal certainty of company policies, and end-user responsibility.

\begin{verbatim}
This reviewer would also suggest to provide more details on 
security aspect of the system. Some hints are provided in 
previous comments. Additionally, as the client/server 
interaction is concerned, what measures and aspects have 
been applied in order to identify and prevent malicious 
parties to manipulate or inject malicious rules in the system? 
By injecting such rules, the complete sub-network of clients 
of the company can be affected. Furthermore, author assumes 
that the clients are non-rooted devices. What happens, and at 
which extent can some rooted device affect local application 
of the system, and also central server and other clients?
\end{verbatim}

\textcolor{blue}{TODO}
% Paloma - Esto creo que lo he respondido antes...
% Antonio - por favor, d� eso mismo al revisor.
% Dir�gelo a la respuesta que sea o repite de otra forma lo mismo.

\begin{verbatim}
For the section 5 and discussion, this reviewer would suggest to
consider providing overview in the form of table(s) or figure,
comparing different aspects of MUSES and other tools and providing
hints/references to further details in this or other works. As there
are many features of MUSES and other tools, employed approaches and
techniques, different coverages and limitations, such overview could
contribute to clarity and position of the proposed solution. 
\end{verbatim}


We agree with the reviewer. Two comparative tables have been included in
section 5 (Table 1 and Table 2). They summarize some of the features of
the presented tools and MUSES regarding the taxonomy we have described in the paper (Section 3); and also their features with respect to licenses, devices and prices.
As it can be seen, MUSES is the most complete considering Table 1. Moreover, having a look at Table 2, all the other systems are proprietary and have a cost, which makes MUSES unique in the sense that, being free software, can be adapted and extended by corporate security departments to the corporate special needs. 


\begin{verbatim}
Overall, this reviewer perceives this work as a good contribution in
presenting and advancing the state-of-the art. It is however suggested
to consider revising this work and take into account provided
suggestions. 
\end{verbatim}
 
We have addressed all the suggestions by the reviewer so we hope the work has been improved enough.

\begin{verbatim}
Specific comments regarding organization and presentation:

1) This work overuses forward slashes in linking words similar in 
concept or meaning. Although they are necessary in some situations, 
forward slashes very often link redundant words and reduce flow and 
readability of the text. They also contribute to the vagueness. I 
would suggest authors to review the text in this concern.
\end{verbatim}

The reviewer is right. We have fixed this in the text omitting their use in those cases.

\begin{verbatim}
2) Authors overuse parentheses to explain details. This hinders 
readability and text flow. The level of details provided and their 
relevance in the context are only partially consistent. On p.14 
users introduce footnotes for additional details. Maybe they 
should consider applying such approach in other parts of the 
document as well.
\end{verbatim}

Following reviewer's suggestion we have rewritten the majority 
of the phrases where parentheses were used, avoiding them for 
the sake of a better text flow and reading.

\begin{verbatim}
3) The structure and flow of the work can be improved.
\end{verbatim}

We have rewritten several parts of the paper and included new sections, subsections, tables and lists in order to improve the structure, contents and fluency of the text. Moreover, some paragraphs have been also rewritten to avoid large phrases, the inadequate use of parentheses, or the misunderstanding of some concepts.
% Antonio - Por favor, completad esta justificaci�n mejor.


We are very grateful to the reviewer for his/her useful comments and hope he/she will be satisfied with the revised work.


%FERGU: Añado las nuevas referencias a partir del fichero bibtex, poned las vuestras!
\bibliographystyle{elsarticle-num}
\bibliography{review_muses}
\end{document}
