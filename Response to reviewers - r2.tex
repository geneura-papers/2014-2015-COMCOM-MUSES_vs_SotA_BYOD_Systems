\documentclass[preprint]{elsarticle}
\biboptions{round, numbers}
\usepackage[latin1]{inputenc}
%\usepackage[T1]{fontenc}
%\usepackage{textcomp}
\usepackage{graphicx}
\usepackage{color}
%\usepackage{setspace}
\usepackage{url}
\usepackage[english]{babel}

\begin{document}

%%%%%%%%%%%%%%%%%%%%%%%%%%%%%%%   TITLE   %%%%%%%%%%%%%%%%%%%%%%%%%%%%%%%

\title{Corporate Security Solutions for BYOD:\\ A Novel User-Centric and Self-Adaptive System: Response to Reviewers\' comments}

\noindent
Dear Sirs,\\

We really appreciate the opportunity for improving our work. Following the reviewers' requests and suggestions we have improved the paper, incorporating all suggestions.  

In the next paragraphs you can find these comments and how they have been addressed, along with the changes done.\\

\noindent
Yours sincerely,\\
The authors.


\section{Editor's comments}
Dear Dr. Mora,

Reviewers' comments on your work have now been received. After analyzing the reviewers' reports and the manuscript, the Guest Editors are recommending that you perform a further MAJOR revision of your manuscript to address the reviewers' comments.

While revising the manuscript please carefully revise the analysis of the state of the art to include all recent and relevant works in the field.
An updated analysis of the state of the art is mandatory for a publication in Computer Communications.

If you decide to revise the work, please submit a list of changes or a rebuttal against each point which is being raised when you submit the revised manuscript.



\textit{Include an overview of all changes made in the paper.}

% -----------------------------------------------------------------------------

\section{Comments by Reviewer \#2}

% Antonio - Voy a sustituir el Verbatim por un cambio de color en las respuestas porque al compilarlo no hace los saltos de l�nea si no se los especificas y el texto se sale del margen.

\begin{quotation}
After reviewing the new version of the document, I admit the authors
have  substantially changed section 3 (state of the art) and they have
improved and extended the introduction and sections 4 (description of
the system) and section 5 (comparison with other systems) I
acknowledge the ideas are really interesting and exciting, but I'm
still not sure if the authors are providing new results not published
in their previous works or just rephrasing their text. 

Since there is an obvious over-the-top quality in their work and some
good ideas, I'll let the editors decide if the contents are novel
enough (compared to other previous works by the same authors) to be
published in the journal.\\ 
\end{quotation}

We would like to thank the reviewer for his/her
  opinion and would lead him/her to have another look at the new
  version, since it has been highly improved.  
As a summary:
%
\begin{itemize}
\item We have described in a more detailed way our previously published papers in this scope, clarifying the differences and contributions of this work with respect to them.
\item Section 2 has been deeply improved adding a much more complete State of the Art in the scope of this work... 
% Antonio - TODO: Fergu, completa esto t� que tienes la idea global de lo nuevo en el SotA, por favor.
%The reviewer can see the set of new references added by the end of this response letter.
New references added after the review can be found at the end of this document
\item The description of MUSES system has been improved. Several points have been clarified and better explained in Section 4. Moreover, a prototype of the system has been tested in an actual company with success, according to the obtained results. This is also commented by the end of that section.
% Antonio - TODO: Revisar que esto se haya dicho/hecho en el paper
\item Section 3 has been rewritten in a big part, including a new taxonomy for the analyzed systems, along with a better comparison with MUSES in Section 5.
% Antonio - TODO: extender esto de la secci�n 3 y 5 un poco m�s.
\item A complete proof-editing process has been done, reviewing the English grammar and rewriting several paragraphs for a better comprehension.
\end{itemize}
% Antonio - TODO: A�adir m�s cosas (medianamente relevantes) que se hayan hecho
%
These changes are better described in the responses to the others reviewers, so we encourage this reviewer to read them to have a better idea of their impact in the new version of the manuscript.

% -----------------------------------------------------------------------------

\section{Comments by Reviewer \#3}

This paper describes MUSES, an adaptable framework for enforcing security policies on mobile devices. This tool relies on Machine Learning and CI techniques to predict policy violations and is self-adaptive on user behaviors.
The idea seems interesting although several technical concerns arise from the reading of this work:
1.      The tools aims at enforcing security policies, however the security response occurs only AFTER a particular pattern is recognized to be potentially malicious. From my understandings, there is no a-priori enforcement except for the prediction based on event logs. This could mean that a security violation could in principle happen and only later recognized. This is not so suitable for a corporate environment with strict security requirements. This statement should be discussed and preferably discredited with experimental results.

\textcolor{blue}{MUSES starts from a defined set of security rules that the CSO must define. Ideally, these rules should deal, in principle, all the potential security incidents that the company aims to detect. MUSES would be able to define new rules (by means of refinement) which could be labelled (i.e. assigned a decision) after the corresponding events have happened. However, it can also infer or create new rules, which the CSO must approve, using computational intelligence techniques. These rules could deal with unexpected situations (not previously happened).
Anyway, the idea is to define a set of `restrictive' rules at the very beginning, able to ensure the system safety in a big amount of situations. Then, MUSES will refine or improve this set, making more specific and optimal rules.}

\textcolor{blue}{Moreover, MUSES offers visual analysis tools to the CSOs, which can aid (or guide) they to detect suspicious or anomalous patterns, and to study how to avoid these patterns becoming malicious to the company.}

\textcolor{blue}{In order to clarify this we have rewritten and completed the second paragraph in section 4.1 (page XX). It is now:}

\textcolor{blue}{\textit{The client is an important part of the a-priori risk treatment, because it both monitors the whole context, and finally acts or warns the user in event of a security violation. 
But security violations are mainly identified by security rules, which is why MUSES needs at least an initial set of rules, defined by the CSO. Then, it makes use of DM and CI techniques to improve or optimise them, enhancing the coverage. 
That is to say, the more sensors are implemented, the more information related to event context can be extracted. Implementing a sensor means to include a programmed method which triggers MUSES to start `listening' to some parts of a device. Then, the initial security rules establish which situations might cause a security violation, and the sensors provide with information which helps the MUSES server to identify events that are alike to the ones that are already known as dangerous.}}

We have also improved this clarification in section 5 (page XX), adding the text:

\textcolor{blue}{\textit{However, MUSES presents a limitation regarding the enhancement of rules, since, in principle, it cannot predict (generate rules for dealing with) unexpected or unknown events which could lead to a security incident. The philosophy is that the initial set of rules, defined by the CSO, should be very restrictive regarding possible unexpected users' behaviours or events, in order to avoid as much security incidents as possible. As the system works, MUSES would be able to define new rules (through refinement) which could get an associated decision after the corresponding events have happened. Thus, this will lead to get eventually an optimal set of security rules.}}

\textcolor{blue}{\textit{In addition, MUSES can also infer or create new rules using computational intelligence techniques. These rules could deal with unexpected situations (not previously happened), but must be previously approved by the CSO. Of course, everything is constrained by the available set of sensors which, in turn, define the possible information that MUSES will analyze and use in the refinement and inference processes.}}

2.      A strong limitation is that an effective enforcement is only possible with MUSES Aware Apps. It is unclear the effectiveness of the approach with commercial applications. In page 14 "MUSES could just detect very specific events and also do very limited actions". With those assumptions, it seems unlikely that a company would adopt MUSES since it is not able to offer an adequate level of security. I think that this is very far from being acceptable for a BYOD solution. Also in this respect, experimental results are needed.

\textcolor{blue}{TODO}

3.      MUSES Sensors are essentially user-space applications that collects logs on specific activities. This aspect really concerns me since a compromised device or compromised sensor apps could vanish all MUSES effort. Authors should deeply discuss this aspect.

\textcolor{blue}{A more detailed definition and description of the sensors has been included. In addition, a discussion on how the case the reviewer mentions could affect the system has been added, as:}

\textcolor{blue}{It is important to note that MUSES does not work as an antivirus itself, but has a sensor which checks if the device has already an antivirus, and raises a security violation alert if not. It might happen then that the user ignores this advice and the device becomes compromised. As it would be explained in the next section, the sensors are constantly sending their monitored values, and therefore patterns are created, stored, and processed on the server. In case the device gets infected with rootkit-type malware \cite{bickford2010rootkits}, a sudden change in the values of the sensors will mean a significant change in the patterns. Then, this change in the patterns will be noticed by the server, which will immediately warn the user, and decrease the device trust value.}

4.      Paper structure is really confusing. Many essential concepts are introduced and explained too late in the paper forcing the reader to go back and forth to understand the architecture (e.g. MUSES aware apps and Data Mining techniques). I suggest authors to deeply revise the presentation of their work.

\textcolor{blue}{TODO}


5.      A discussion on research activities on BYOD is completely missing. This is unacceptable for a solution that aims at solving BYOD security issues. Just to cite a few:
a.      Gessner, Dennis, et al. "Towards a User-Friendly Security-Enhancing BYOD Solution." NEC Technical Journal 7.3 (2013): 113.
b.      Samaras, Vasileios. A BYOD Enterprise Security Architecture for accessing SaaS cloud services. Diss. TU Delft, Delft University of Technology, 2013.
c.      Armando, Alessandro, et al. "Enabling BYOD through secure meta-market." Proceedings of the 2014 ACM conference on Security and privacy in wireless and mobile networks. ACM, 2014.


\textcolor{blue}{The reviewer is right. Although several works have been cited in other sections of the work, a proper discussion on research on BYOD was necessary. Therefore, we have analysed the papers suggested by the reviewer, along with several new relevant ones. These new references also have been also cited in other sections of the work to justify MUSES features; for example, the set of best practices for BYOD security proposed by H. Romer. These works have been also considered to propose the new taxonomy included in the paper (Section X, Page XX). These references are \cite{Garba15organisational,Scarfo12survey,Gessner13userfriendly,Armando14metamarket,Haejung12Door,Samaras13SaaS,Miller12Privacy,Romer14BestPractices}, and can be found at the end of this document.}

6.      The experimental section is completely missing. Is MUSES able to identify security violations? Which are the computational cost for both the company (server) and users (mobile devices)? Are there any limitations?


\textcolor{blue}{We agree with what the reviwer remarks, and so we have included, at the end of the Introductiono section, the ratio of security violations detected by the system in a prototype trials period. Also, a little description of the conditions of the trials has been included, for we consider that a deep description of these trials is not in the scope of the paper.}

Under such considerations I recommend authors for a *STRONG* major revision to cover the aforementioned points.

\textcolor{blue}{The work has been deeply improved following the reviewer's suggestions. We would like to thank he/she for his/her careful revision and interesting comments.}


% -----------------------------------------------------------------------------

\section{Comments by Reviewer \#4}

\begin{quotation}
In this work authors present state-of-the art in BYOD solutions, and
introduce a new solution, MUSES, a result of an EU FP7 project. 

General and specific comments:

The abstract of the paper should be improved. Sentence on page 1,
lines 23-25 is unclear. It suggests that the BYOD solutions appeared
in the market with the purpose to adapt to it in a secure way. 
\end{quotation}

The abstract has been changed and that confusing sentence has been
eliminated. Now we think it is clear, as intended, the motivation,
context and focus of the paper. 

\begin{quotation}
The description of IBM products, in section 3.1, does not provide
clear distinction between various service types provided by IBM. It
furthermore lacks focused coverage of services relevant for this
work. 

In analysis of IBM offer, authors refer to general web site for mobile
security products and consumer guide/whitepaper that explores the
general problem of BYOD. They consider services that are not purely
related to BYOD and MUSES. For instance, it is not clear how
"Enterprise Wireless Networks" (page 6, lines 21-26) fit into
comparison, and how establishment of secure connections can be related
to MUSES's adaptability. On the other hand, the products such as
MobileFirst and MaaS360 are omitted. The suggestion in this regard is
to consider rewriting this part, including updated references and
current tools in comparison. 
\end{quotation}


Following reviewer's suggestions, this section has been rewritten, and
the references mentioned by the referee, including new products, have
been considered in the study. Tools have been described 
(end of Section 3.1, in page XX), and compared to MUSES in Section 5.
%FERGU: PONED ESTO MÁS BONICO!
% Antonio - ahora est� un poco mejor, pero no s� si m�s bonico. Fergu debes justificar el por qu� se han puesto los tres servicios de IBM, ya que el revisor dice que algunos no est�n relacionados con el tema del trabajo... t� que sabes de esto de IBM. ;D
% Fergu, tambi�n falta la comparativa con MUSES de esas herramientas (Secci�n 5)...
% [Pedro] he cambiado un poco la redacci�n   ;)

\begin{quotation}
In section 3.3 (p.7, sentence in l.18-19) there is provided a
statement about success of two solutions, however without referencing
relevant sources or providing details. It is furthermore questionable
whether this statement is relevant for the scope of this work at all.
\end{quotation}

\textcolor{blue}{TODO}

\begin{quotation}
Section 3.6.1 covers WSO2, an open source tool that should be freely
available for evaluation and comparison. However, in review, this tool
has been only roughly mentioned. The context and relevance to the work
are omitted, as well as features and comparison with MUSES. It is not
clear therefore how this tool and MUSES differ. 
\end{quotation}

\textcolor{blue}{TODO}


Section 3.6.4 (p10, 44) states "are Worx-enable are capable". The meaning of this part is not clear, nor to which concept Worx refers.

\textcolor{blue}{TODO}

Basically, in section 3 the following possibilities for improvement are identified:

 1) Explain why particular solutions are chosen for inclusion in review and what is their relevance? Did authors consider other (possible) solutions relying on DM and CI?

\textcolor{blue}{TODO}

 2) Considering variety of BYOD tools and their different features, authors might consider to introduce or reuse a taxonomy or classification of those approaches, and state the place of MUSES there (or define a new subcategory, if deemed necessary)

\textcolor{blue}{Following the reviewer's suggestion we have introduced a taxonomy for the analyzed systems, according to the works that we have revised in the state of the art section. This taxonomy is described at the beginning of Section 3 (second paragraph, page XX).}
% Antonio - completar con el n�mero de p�gina final
% Antonio - TODO: decir si MUSES y el resto de sistemas se han encuadrado dentro de esa taxonom�a

 3) Some presented solutions miss their relationship to MUSES (coverage, features, comparison). This point also refers to inconsistent coverage amongst solutions and weak arguments favouring MUSES. Either coverage should be more consistent by focusing on common (present or missing) aspects, or it should be elaborated why such coverage is missing. Also see the comment for discussion part bellow.

\textcolor{blue}{TODO - justificar que todos los sistemas no tienen las mismas propiedades, pero s� la misma filosof�a}

On p.12, l.9-12 authors state that MUSES can be installed in every mobile or portable device, independently of the operating system and device type. This is also derived from the title "Multiplatform Usable Endpoint Security System".
The authors however do not provide additional details on how is this achieved in the practice and are there any limitations in regards to that. How they managed to provide the same system for Android, Windows, iOS, PCs, without limitations? How MUSES manage the variety of OS versions and underlying platform features provided to the app? This feature is important differentiator against some other solutions and in opinion of this reviewer should be elaborated in more detail.

\textcolor{blue}{TODO - describir las limitaciones de MUSES en cada sistema}


When stating (p.13, l.15-24) that MUSES relies on AES based symmetric encryption, does it mean that information flows in MUSES client internally, as well as client-server communication, are encrypted with the same key? Does MUSES uses client specific and/or temporal keys? What measures are taken to protect this key from malicious applications both on client and server side? How and where is this key generated? What is lifetime of the key? In my opinion this part should be reworked, providing more detailed description of security aspects of their approach. Also please see security related comment bellow.

\textcolor{blue}{TODO}

On p.14, l.16-28 authors introduce the concepts of MUSES Aware App and MUSES API. In my opinion they should provide more details on integration options and necessary effort to make an application MUSES Aware. This is a crucial aspect of their approach as this requirement and related complexity might hinder the adoption of the solution.

\textcolor{blue}{TODO}

On p.17, l.7-9 authors state that the data sets can be huge, leading to necessity to apply Big Data processing methods. They however do not provide at least an indication of the data amount/complexity that is gathered and processed by the system, and at what rate it grows. Is it really necessary to employ Big Data processing?

\textcolor{blue}{TODO - Incluir justificaci�n de Big Data Processing considerando el n�mero de eventos generados en las Trials y extrapol�ndolo a muchos meses (al menos los 6 del hard limit)}.

On p.18, for two bullets announced under l.46-47 there is not clear if they are included in the refinement process or not. In the later case, it is suggested to explain why they are not present in the processing. Maybe these bullets could be moved into part relating to further work.

\textcolor{blue}{TODO - no s� a qu� 'bullets' se refiere}

From the presented concepts behind MUSES and its flows, it is not completely clear how it approaches "newly discovered vulnerabilities or threats" (p.20 l.27-28). It is hence suggested to clarify this statement and provide explanation to which extent is rule refinement and adaptation used to discover and address new vulnerabilities and threats in the software and systems.

\textcolor{blue}{TODO}

Similar applies to human-computer interaction (p.20 l.34). Although some details are provided in l.43-45 on the same page, in opinion of this reviewer, the statement about progress beyond the state of the art should be backed with more (stronger) details and data.

\textcolor{blue}{TODO}

This reviewer would also suggest to provide more details on security aspect of the system. Some hints are provided in previous comments. Additionally, as the client/server interaction is concerned, what measures and aspects have been applied in order to identify and prevent malicious parties to manipulate or inject malicious rules in the system? By injecting such rules, the complete sub-network of clients of the company can be affected. Furthermore, author assumes that the clients are non-rooted devices. What happens, and at which extent can some rooted device affect local application of the system, and also central server and other clients?

\textcolor{blue}{TODO}

\begin{quotation}
For the section 5 and discussion, this reviewer would suggest to
consider providing overview in the form of table(s) or figure,
comparing different aspects of MUSES and other tools and providing
hints/references to further details in this or other works. As there
are many features of MUSES and other tools, employed approaches and
techniques, different coverages and limitations, such overview could
contribute to clarity and position of the proposed solution. 
\end{quotation}

We agree with the reviewer. A comparative table has been included in
the section 5 (Table 1), summarizing some of the features of
the presented systems and those of MUSES system. As it can be seen,
all other systems are proprietary and have a cost, which makes MUSES
unique in the sense that, being free software, can be adapted and
extended by corporate security departments to the corporate special
needs. 

% Antonio - TODO: M�s Tablas comparativas. �Antares, puedes hacerlas?
% Antonio - �quiz� una tabla con lo de la taxonom�a y en qu� categor�a entra cada sistema?

\begin{quotation}
Overall, this reviewer perceives this work as a good contribution in
presenting and advancing the state-of-the art. It is however suggested
to consider revising this work and take into account provided
suggestions. 
 
\textcolor{blue}{We have addressed all the suggestions by the reviewer so we hope the work has been improved enough.}

Specific comments regarding organization and presentation:

1) This work overuses forward slashes in linking words similar in concept or meaning. Although they are necessary in some situations, forward slashes very often link redundant words and reduce flow and readability of the text. They also contribute to the vagueness. I would suggest authors to review the text in this concern.

\textcolor{blue}{The reviewer is right. We have fixed this in the text omitting their use in those cases.}

2) Authors overuse parentheses to explain details. This hinders readability and text flow. The level of details provided and their relevance in the context are only partially consistent. On p.14 users introduce footnotes for additional details. Maybe they should consider applying such approach in other parts of the document as well.

\textcolor{blue}{Following reviewer's suggestion we have rewritten the majority of the phrases where parentheses were used, avoiding them for the sake of a better text flow and reading.}

3) The structure and flow of the work can be improved.

\textcolor{blue}{We have rewritten several parts of the paper and included new sections, subsections, tables and lists in order to improve the structure, contents and fluency of the text.}
% Antonio - Por favor, completad esta justificaci�n mejor.


\textcolor{blue}{We are very grateful to the reviewer for his/her useful comments and hope he/she will be satisfied with the revised work.}


%FERGU: Añado las nuevas referencias a partir del fichero bibtex, poned las vuestras!
\bibliographystyle{elsarticle-num}
\bibliography{review_muses}
\end{document}
